
\documentclass[aip, reprint]{article}

\usepackage{graphicx}
\usepackage{makeidx}

\usepackage{hyperref}

%test commit

\usepackage[utf8]{inputenc}
\usepackage[english]{babel}
\usepackage{fancyhdr}
 
\pagestyle{fancy}
\fancyhf{}
\fancyhead[LE,RO]{Giuseppe De Martino}
\fancyhead[RE,LO]{CDT-eBankingServices}



\begin{document}


\begin{titlepage}
	\centering
	\includegraphics[width=12cm]{logo_unimi.jpg} 

	Progetto di Cittadinanza digitale e tecnocivismo\\
	{\bfseries\Large
       		\textbf{ Analisi dell’evoluzione della user relationship del cittadino con i servizi bancari}\\
	}    
	di Giuseppe De Martino\\
	\vskip13cm
	\includegraphics[width=2cm]{by-nc-nd.png} 
\end{titlepage}





\newpage
\tableofcontents
\newpage

%%%%%%%%%%%%%%%%%%%%%%%%%%%%%%%%%%%%%%%%%%%%%
%															 %
%															 %
%							ABSTRACT							 %
%															 %
%															 %
%%%%%%%%%%%%%%%%%%%%%%%%%%%%%%%%%%%%%%%%%%%%%


\begin{quotation}
\textbf{Abstract}\\
Il presente documento illustra un progetto di “Cittadinanza Digitale e Tecnovicismo” relativo al cambiamento del rapporto tra cittadino e servizi bancari, come conseguenza del progresso tecnologico e dell’introduzione di nuove tecnologie.
Il documento conterrà le seguenti sezioni: un capitolo introduttivo in cui verrà fornita una breve storia dei servizi bancari prima dell’avvento dei calcolatori.
Di seguito vi sarà una sezione per ogni una delle tecnologie che hanno segnato l’evoluzione dei servizi: la prima sezione introduce il suo utilizzo e delle tecnologie derivate da esso; la seconda sezione parla di interne e dei primi metodi di pagamento; la terza sezione riguarda il web e della nascita delle prima home banking e dei problemi legati alla privacy e sicurezza; la quarta sezione spiega come il web 2.0 evolve i servizi di home banking e dei vari meccanismi di sicurezza adottati tra varie banche; la quinta sezione parla dei smartphone, delle prime app bancarie e dei problemi legati all’aspetto strumentale cognitivo; la sesta sezione parla del bitcoin e dei possibili vantaggi e svantaggi che possono nascere tra banche e/o cittadini; l’ultima sezione è dedicata all’utilizzo dei social network da parte delle banche e della partecipazione al miglioramento dei servizi grazie ad esso.
A fine documento verrà fornita una matrice a doppia entrata costruita sulla base di una raccolta di molteplici fonti bibliografiche.
Le due entrate della matrice si riferiscono a:
\begin{enumerate}

  \item Evoluzione dei servizi bancari / tecnologici: elenco delle milestone che hanno caratterizzato il progresso (insieme di tecnologie sia hardware che software);
  \item Impatto sul cittadino: analisi dell’effetto delle novità tecnologiche sulle aree rilevanti della vita del cittadino.


\end{enumerate}



\end{quotation}

\newpage

%%%%%%%%%%%%%%%%%%%%%%%%%%%%%%%%%%%%%%%%%%%%%
%															 %
%															 %
%							INTRO							 %
%															 %
%															 %
%%%%%%%%%%%%%%%%%%%%%%%%%%%%%%%%%%%%%%%%%%%%%

\textbf{Introduzione}\\
Questo progetto si inserisce all’interno dell’arcobaleno dei diritti di cittadinanza digitale, in particolare, nel livello 03: diritto del cittadino ad usufruire dei servizi online, pubblici o privati.
L’ambito dei servizi preso in considerazione, nello specifico, è quello dei servizi bancari, che con l’avanzare del tempo e del progresso tecnologico, diventano sempre più ampi e pervasivi.
Prima dell’introduzione dei calcolatori, le tempistiche per eseguire ogni singola operazione bancaria erano piuttosto lunghe ed inoltre i rapporti tra le banche e i clienti erano monocanale. L’unico canale consisteva nella filiale, in cui il cittadino doveva recarsi di persona e in orari prestabiliti.
Le prime comunicazioni tra banche, oppure tra uffici centrali e filiali della stessa banca, avvenivano tramite posta tradizionale. A partire dal 1846 viene introdotto il telegrafo come mezzo di comunicazione. 
E al fine di abbattere i tempi di comunicazione, le banche inglesi e quelle americane, a partire dal 1886, iniziano a servirsi del cavo transatlantico (cavo sottomarino costruito e posato sul fondo dell’oceano), con una conseguente riduzione dei tempi di un’operazione standard da 6 settimane ad un giorno. 
Successivamente, viene introdotto il telefono ma in sostanza le prime innovazioni tecnologiche sono state sfruttate per migliorare i processi interni alle organizzazioni bancarie, specialmente quelli di comunicazione tra uffici centrali e filiali.
Solo con la banca telefonica, si assiste ad un primo avvicinamento al cliente, ma fino all’introduzione dello sportello bancario è l’unica alternativa per il cittadino alla filiale. 
Il vero obiettivo di miglioramento della relazione banca-cliente viene preso in considerazione soltanto quando le banche capiscono che il fenomeno dell’uso di un pc connesso a internet permea il tessuto sociale. E con l’arrivo del web tutta l’industry bancaria comincia a digitalizzare prodotti e servizi per soddisfare i nuovi consumatori.\\


%%%%%%%%%%%%%%%%%%%%%%%%%%%%%%%%%%%%%%%%%%%%%
%															 %
%															 %
%							MOTIVAZIONI						 %
%															 %
%															 %
%%%%%%%%%%%%%%%%%%%%%%%%%%%%%%%%%%%%%%%%%%%%%

\textbf{Motivazioni}\\
La scelta, di prendere in esame l'evoluzione dei servizi bancari e il loro impatto nei confronti del cittadino, nasce sia dalle esperienze lavorative acquisite nei miei ultimi due anni lavorativi sia dal fatto che un quadro così ampio 

\newpage

%%%%%%%%%%%%%%%%%%%%%%%%%%%%%%%%%%%%%%%%%%%%%
%															 %
%															 %
%							COMPUTER							 %
%															 %
%															 %
%%%%%%%%%%%%%%%%%%%%%%%%%%%%%%%%%%%%%%%%%%%%%

\section{Anni ’60: il computer nelle banche}
La prima banca ad utilizzare il calcolatore è stata la Bank of America introducendo verso la fine degli anni ’50 l’ERMA Electronic Recording Method of Accounting, il suo utilizzo è quello di velocizzare un numero limitato di servizio di back office. Solo con l’introduzione dello sportello ATM le banche migliorano il servizio fornito al cittadino.\\
\subsection{Breve storia dei calcolatori nelle banche}
Il primo calcolatore ad essere stato impiegato in una banca fu ERMA, acronimo di Electronic Recording Method of Accounting.
ERMA è un progetto portato a termine dai ricercatori della Stanford ed ideato per la Bank of America nel tentativo di informatizzare il settore bancario. ERMA si occupava di informatizzare il tutta la parte manuale relativa ai controlli e gestione degli conti correnti. 
I ricercatori della Stanford hanno anche inventato MICR (Magnetic Ink Character Recognition) come parte del ERMA. MICR permette ai computer di leggere i numeri speciali posti sul fondo degli assegni ed ha consentito il monitoraggio informatizzato e contabilizzazione delle operazioni di controllo. [8]

DA RICERCARE Primo computer in italia nelle banche

A parte la velocizzazione dei tempi delle operazioni non vi sono particolari novità dal punto di vista dei servizi al cittadino. 
A partire dall’introduzione dell’ERMA il tema dell’automazione/computerizzazione dei servizi tramite sistemi di data processing diventa centrale. Nonostante ciò le banche non erano padrone delle tecnologie applicate e soprattutto, essendo tecnologie pioneristiche non era ancora sorta la necessità di formulare leggi sul “legale utilizzo e trattamento dei dati personali”

\subsection{Miglioramento del servizio di front-office}
Per quanto riguarda i servizi bancari lo sportello Atm rappresenta il primo canale diretto tra cittadino e automazione, in quanto per la prima volta i servizi vengono erogati tramite macchina senza operatore.
La primissima introduzione dello sportello ATM avvenne nel 1920, ma essendo stata presentata come un’innovazione in aggiunta al personale bancario non ebbe successo. Venne poi rintrodotto nel 1960 quando un cittadino scozzese, Shepherd-Barron, non poté usufruire dei servizi bancari perché si era recato in filiale durante l’orario di chiusura: da qui la sua idea di creare una macchina sempre in funzione che potesse dispensare contanti continuativamente. La macchina venne installata in una filiale di Barclay a nord di Londra.
Era dotata anche di un sistema di riconoscimento degli assegni tramite Carbonio 14 (sostanza radioattiva ma senza conseguenze sull’utilizzatore finale).

Nel 1965 venne sviluppata l'idea del codice numerico di sicurezza (PIN) elaborata dall'ingegnere inglese James Goodfellow,  titolare di alcuni brevetti in materia, tuttavia il modello inaugurato nel 1967 accettava soltanto voucher monouso, che venivano trattenuti dalla macchina.

Contemporaneamente in USA un ingegnere americano, Don Wetzel, inventò il primo Docuteller ATM che era in grado di riconoscere anche le carte di plastica. [12][13][14]

In Italia compare nel 1976, a Ferrara: la Cassa di Risparmio di Ferrara fu la prima banca italiana a installarlo.[15]

Fu però solo negli anni 80 che vi fu una massiva diffusione dello sportello ATM, come noi lo intendiamo oggi.

\subsection{Confronto tra USA e EU}

\newpage
%%%%%%%%%%%%%%%%%%%%%%%%%%%%%%%%%%%%%%%%%%%%%
%															 %
%															 %
%							INTERNET							 %
%															 %
%															 %
%%%%%%%%%%%%%%%%%%%%%%%%%%%%%%%%%%%%%%%%%%%%%
\section{Anni ’70: Internet e le transazioni “in rete”}
Con Internet le banche colgono l’occasione per migliorare la velocità di comunicazione tra uffici centrali e filiali. 
Con internet viene introdotto il POS e gli accordi tra clienti e banche si evolvono in relazione a questa tecnologia

\subsection{Evoluzione dei metodi di pagamento} 
\subsection{Storia del Point of sale (pos)}
\subsection{Pos –government: le nuove regole}
\subsection{Confronto tra USA e EU}

\newpage
%%%%%%%%%%%%%%%%%%%%%%%%%%%%%%%%%%%%%%%%%%%%%
%															 %
%															 %
%							WEB 								 %
%															 %
%															 %
%%%%%%%%%%%%%%%%%%%%%%%%%%%%%%%%%%%%%%%%%%%%%
\section{Anni '90: il web}
Nascono nuovi orizzonti per il cittadino: il computer è un mezzo per ottenere informazioni, oltre che per lo svago. Sempre più persone iniziano a capire l’utilità del computer, e le banche approfittano di questo momento storico per creare la gamma dei loro servizi informativi al cittadino e nascono i primi home banking.
\subsection{La nascita dei primi servizi di e-banking}
\subsection{Primi problemi di sicurezza delle informazioni}



\newpage
%%%%%%%%%%%%%%%%%%%%%%%%%%%%%%%%%%%%%%%%%%%%%
%															 %
%															 %
%							WEB2.0							 %
%															 %
%															 %
%%%%%%%%%%%%%%%%%%%%%%%%%%%%%%%%%%%%%%%%%%%%%
\section{Anni 2000: web 2.0}
Le banche ampliano la gamma di servizi disponibili online per il cittadino, dalla semplice ricarica telefonica al trading online (TOL).
Migliora la qualità dei servizi e la sicurezza delle transazioni. 
Nascono le prime banche che forniscono solo servizi online.

\subsection{Le banche online: miglioramento dell’aspetto contrattuale tra banca e cittadino}
\subsection{Miglioramento della sicurezza delle informazioni}
\subsection{Confronto tra le tecniche di sicurezza adottate da diverse banche}


\newpage
%%%%%%%%%%%%%%%%%%%%%%%%%%%%%%%%%%%%%%%%%%%%%
%															 %
%															 %
%						SMARTPHONE							 %
%															 %
%															 %
%%%%%%%%%%%%%%%%%%%%%%%%%%%%%%%%%%%%%%%%%%%%%
\section{2010: Smartphone}
La banca coglie l’occasione per fornire al cittadino i suoi servizi tramite questo device. Non è più il cittadino che si reca in filiale o si posiziona davanti al pc ma sono i servizi bancari che diventano ‘mobili’ e lo seguono ovunque.

\subsection{Le app di home banking}
\subsection{Aspetto cognitivo-strumentale: problematiche del digital divide}
\subsection{Servizio one-click}


\newpage
%%%%%%%%%%%%%%%%%%%%%%%%%%%%%%%%%%%%%%%%%%%%%
%															 %
%															 %
%							BITCOIN							 %
%															 %
%															 %
%%%%%%%%%%%%%%%%%%%%%%%%%%%%%%%%%%%%%%%%%%%%%
\section{Giorni nostri: Bitcoin}
Il Bitcoin è uno strumento a cui le banche prestano particolare attenzione perché con questa nuova introduzione il concetto di banca è arrivato ad un bivio cruciale: evolversi o morire.
Con la logica del Bitcoin ognuno è la banca di sé stesso 

\newpage
%%%%%%%%%%%%%%%%%%%%%%%%%%%%%%%%%%%%%%%%%%%%%
%															 %
%															 %
%						SOCIAL NETWORK							 %
%															 %
%															 %
%%%%%%%%%%%%%%%%%%%%%%%%%%%%%%%%%%%%%%%%%%%%%
\section{Giorni nostri: Social Network}
Con i social comincia la ‘partecipazione’ del cittadino (indipendentemente dal fatto che sia cliente o meno) alla banca. La comunicazione diventa bidirezionale: da banca a cittadino e, soprattutto, da cittadino verso la banca.
Lo strumento è utile alla banca per una profilazione più dettagliata delle persone interessate ai suoi prodotti e servizi.



\newpage
%%%%%%%%%%%%%%%%%%%%%%%%%%%%%%%%%%%%%%%%%%%%%
%															 %
%															 %
%						CONCLUSIONI							 %
%															 %
%															 %
%%%%%%%%%%%%%%%%%%%%%%%%%%%%%%%%%%%%%%%%%%%%%
\textbf{Conclusioni}\\



\newpage
%%%%%%%%%%%%%%%%%%%%%%%%%%%%%%%%%%%%%%%%%%%%%
%															 %
%															 %
%						BIBLIOGRAPHY							 %
%															 %
%															 %
%%%%%%%%%%%%%%%%%%%%%%%%%%%%%%%%%%%%%%%%%%%%%
\begin{thebibliography}{9}

\bibitem{lamport94}
	LeslieSupriya Singh, Margaret Jackson, Jennie Beekhuyzen, Anuja Cabraal,
	\emph{The bank adn i: privacy, banking ad life stage}.

\bibitem{lamport94}
	Leslie Yi-Jen Yang,
	\emph{The security of elettronic banking},
	Adelphi, MD. 20783.

\bibitem{lamport94}
	Georgios Samakovitis,
	\emph{U.K. banking expert as decision-markes: a historial view on banking tenchologies}
	University of Greoniwich.

\bibitem{lamport94}
	Missouri State University
	\emph{How willtecnology impact Banks, Fabian Forlant},
	

\bibitem{lamport94}
	Avrid O.I. Offmann, 
	\emph{The impact of fraud prevention on bank-customer relationships: an empirical investigation in retail banking},
	Maastricht University and Cornelia Birnbrich.


\bibitem{lamport94}
	Alhaji Abubakar Aliyu, Rosmaini Bin HJ 
	\emph{The impact of information and communication technology on banks’performance and customer service delivery in the banking industry}
	TasminUniversity Tun Hussein Onn Mlayesia.

\bibitem{lamport94}
	S.T. Akinyele, K. Olorunleke,
	\emph{Technology and Service quality in the banking industries: an empirical study of various factor in eletronic banking services}
	Coveant University.

\bibitem{lamport94}
	\url{http://inventors.about.com/library/inventors/bl_ERMA_Computer.htm}

\bibitem{lamport94}
	\url{http://about.bankofamerica.com/en-us/our-story/bank-of-america-revolutionizes-industry.html}
\bibitem{lamport94}
	\url{http://en.wikipedia.org/wiki/Electronic_Recording_Machine,_Accounting}
\bibitem{lamport94}
	\url{http://www.historyofinformation.com/expanded.php?id=992}

\bibitem{lamport94}
	\url{http://www.engineersgarage.com/invention-stories/atm-history#}
\bibitem{lamport94}
	\url{http://inventors.about.com/od/astartinventions/a/atm.htm}
\bibitem{lamport94}
	\url{http://inventors.about.com/od/astartinventions/a/atm_2.htm}
\bibitem{lamport94}
	\url{http://it.wikipedia.org/wiki/Automated_Teller_Machine}



\end{thebibliography}



\end{document}

